\documentclass[11pt,a4paper]{article}
\usepackage[utf8]{inputenc}
\usepackage[T1]{fontenc}
\usepackage{amsmath,amssymb}
\usepackage{geometry}
\geometry{margin=1in}
\usepackage{graphicx}
\usepackage{hyperref}
\usepackage{xcolor}
\usepackage{booktabs}
\usepackage{caption} % Added for better image captions

\title{An Introduction to Machine Learning}
\author{Author Name}
\date{\today}

\begin{document}

\maketitle

\section{Introduction to Machine Learning}

Machine Learning (ML) is a powerful subfield of Artificial Intelligence (AI) that empowers computer systems to learn from data, identify patterns, and make decisions or predictions with minimal human intervention. Unlike traditional programming, where explicit rules are coded, ML algorithms are designed to adapt and improve their performance over time as they are exposed to more data. This paradigm shift enables machines to tackle complex problems that are difficult or impossible to solve with fixed, rule-based logic.

The core idea behind machine learning is to build models that can generalize from observed examples. For instance, instead of programming a computer to recognize a cat by listing all its features, an ML model can be trained on thousands of images labeled as "cat" or "not cat." Through this training, the model learns to identify the underlying patterns and characteristics associated with cats, allowing it to correctly classify new, unseen images.

\begin{figure}[htbp]
    \centering
    \includegraphics[width=0.7\textwidth]{example-image-a}
    \caption{A conceptual representation of the machine learning process: data input, model training, and prediction output.}
    \label{fig:ml_process}
\end{figure}

The applications of machine learning are vast and ever-expanding, profoundly impacting various aspects of modern life. From personalized recommendations on streaming platforms and e-commerce websites to fraud detection in banking, autonomous vehicles, medical diagnosis, natural language processing, and scientific research, ML is at the forefront of technological innovation. Its ability to process and derive insights from massive datasets makes it an indispensable tool in the era of big data.

\subsection{Types of Machine Learning}
Machine learning algorithms are broadly categorized into several types based on the nature of the training data and the learning process:

\begin{itemize}
    \item \textbf{Supervised Learning:} This is the most common type of ML, where the model learns from labeled data. Each training example consists of an input and a corresponding correct output. The algorithm's goal is to learn a mapping function from inputs to outputs. Examples include image classification, spam detection, and regression tasks like predicting house prices.
    \item \textbf{Unsupervised Learning:} In contrast to supervised learning, unsupervised learning deals with unlabeled data. The algorithms try to find hidden patterns, structures, or relationships within the data on their own. Clustering (e.g., customer segmentation) and dimensionality reduction are primary examples.
    \item \textbf{Reinforcement Learning:} This type of learning involves an agent interacting with an environment to learn optimal behavior. The agent receives rewards or penalties for its actions, learning through trial and error to maximize cumulative reward. It's often used in robotics, game playing (e.g., AlphaGo), and control systems.
\end{itemize}

\begin{figure}[htbp]
    \centering
    \includegraphics[width=0.6\textwidth]{example-image-b}
    \caption{Illustrating the three main paradigms of machine learning: Supervised, Unsupervised, and Reinforcement Learning.}
    \label{fig:ml_types}
\end{figure}

The rapid advancements in computational power, the availability of large datasets, and the development of sophisticated algorithms have propelled machine learning into a transformative technology. As we continue to generate more data, the capabilities and impact of machine learning are only expected to grow, offering innovative solutions to complex challenges across all domains.

\end{document}